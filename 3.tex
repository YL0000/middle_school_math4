
\chapter{多项式的基础理论}
在初中我们已经学习过多项式及其四则运算,并着重学习了一元多项式的带余除法、余式定理和多元多项式的乘法公式,因式分解。这都是重要的基础知识,在数学和实际中都有广泛的应用,本章将从理论和应用上对多项式的基础知识作进一步的研究、提高,我们研究的重点仍然是一元多项式。

\section{多项式及其代数运算}
多项式的概念我们并不陌生,尤其是一元多项式,每个人都能举出不少例子。它的四则运算也会用各种方法进行。总括我们已经学过的知识,可以一般地系统整理如下:

\subsection{多项式的概念}
\begin{blk}{定义1}
    形如$a_nx^n+a_{n-1}x^{n-1}+\cdots+a_1x +a_0$
的式子,叫做$x$的一元多项式(简称多项式)。其中,$a_i$ $(i=0, 1, 2,\ldots,n)$是已知实数,$n$是已知非负整数。
\end{blk}
.
一元多项式一般简记为$f(x)$或$g(x)$等,即
$$f (x) =a_nx^n+a_{n-1}x^{n-1}+\cdots+a_1x +a_0$$

在多项式$f(x)$中,$a_ix^i$ $(i=0, 1, 2,\ldots,n)$叫做$f(x)$的$i$次项,$a_i$叫做$i$次项的系数;当$a_i\ne 0$时,多项式$f(x)$称为一元$i$次多项式,并把它的次数记作${\rm deg} f(x)=i$。

特别地,当$n=0$时,多项式成为
$f (x) =a$,
这时,若$a_0\ne 0$, 就叫做零次多项式;若$a_0=0$就叫做零多项式,它的次数不定义。

例如,
$f_1(x)=7x^3-1$叫做一元三多项式,$f_2(x)=-5$叫做零次多项式,$f_3(x)=0$叫做零多项式,它不定义次数。

\begin{rmk}
    在初中我们把多项式中的字母。称为未知数,也称为元。现在我们还可以用函数的观点把它称为自变数,甚至可以更一般地称为不定元。它和数作运算时满足数系运算通性,即满足加法和乘法的结合律、交换律以及乘法对加法的分配律;同时,零与1的运算特性、指数运算律仍然适合。
\end{rmk}

这样一来,任何一个$n$次多项式,经过整理合并同类项,总可以写成标准形式
\begin{equation}
f(x)=a_nx^n+a_{n-1}x^{n-1}+\cdots+a_1x +a_0\quad (a_n\ne 0)
\end{equation}
或者
\begin{equation}
    f(x)=a_0+a_1x+\cdots+a_{n-1}x^{n-1}+a_nx^n\quad (a_n\ne 0)
\end{equation}
其中(3.1)称为多项式$f(x)$的降幂标准式,(3.2)称为多项式$f(x)$的升幂标准式。

例如,多项式$g (x) =5x-7x^2+13x^4-8x-x^3+10x^2-1$
经过整理后,可以写成降幂或升幂两种标准形式
\[g (x) =13x^4-x^3+3x^2-3x-1\]
或者
\[g (x) =-1-3x+3x^2-x^3+13x^4\]

\begin{blk}{定义2}
如果用一个已知数$b$去代替多项式中的元$x$, 就得到 
\[f(b)=a_nb^n+a_{n-1}b^{n-1}+\cdots+a_1b +a_0\]
那么,数$f(b)$就叫做当$x=b$时$f(a)$的值。
\end{blk}

\begin{example}
    已知$f(x)=a_3x^3+a_2x^2+a_1x+a_0\quad (a_3\ne 0)$, 试求$f (0)$, $f (1)$, $f (-1)$, $f (m)$。 
\end{example}


\begin{solution}
\[\begin{split}
    f (0) &=a_0\\
    f (1) &=a_3+a_2+a_1+a_0\\
    f (-1) &= -a_3+a_2-a_1+a_0\\
    f (m) &=a_3m^3+a_2m^2+a_1m+a_0
\end{split}\]
\end{solution}

\begin{example}
    已知$f(x)=x^2+2x+8$, 求$f(-x)$, $f(x+1)$。
\end{example}

\begin{analyze}
    由于$-x$, $x+1$都不是已知数,因而所求的$f(-x)$, $f(x+1)$也不会是一个已知数值,严格地说题目已不是求值问题。但我们可以理解为要求用$-x$与$x+1$分别代替$f(x)$中的$x$所得的新多项式。实际上就是换元,其运算程序与求多项式的值是相同的。
\end{analyze}

\begin{solution}
\[\begin{split}
    f(-x)&=(-x)^2+2(-x)+3=x^2-2x+3\\    
    f(x+1)&=(x+1)^2+2(x+1)+3\\
    &=x^2+2x+1+2x+2+3=x^2+4x+6
\end{split}\] 
\end{solution}

\begin{blk}{定义3}
    两个多项式
\[\begin{split}
f (x) &=a_nx^n+a_{n-1}x^{n-1}+\cdots +a_1x+a_0\\
g (x) &=b_nx^n+b_{n-1}x^{n-1}+\cdots+b_1x+b_0.    
\end{split}\]
如果它们的各同次项系数对应相等,即$a_k=b_k$ ($k$为非负整数)我们就说这两个多项式相等,记作$f(x)=g(x)$。
\end{blk}


不难知道,两个非零多项式相等的必要条件是它们的次数相等,即如果$f(x)=g(x)$, 那么${\rm deg}f(x)={\rm deg}g(x)$.

应该指出,如果把多项式看作一个函数式,那么两个多项式相等就可推出当自变数取任意允许值时,两个多项式的值都是相等的。在这种意义下,我们把两个多项式相等也可以说成“恒等”。

\begin{example}
已知多项式
\[f (x) =x^3+ (a+3) x^2+bx-1\]
与多项式
\[g (x) =x^3- (1-b) x^2+ (10-a) x-1\]
相等,试求$a,b$的值。    
\end{example}

\begin{solution}
设$f(x)=g(x)$, 且都已是降幂标准式,所以它们的各同次项系数对应相等。因而有    
\[\begin{cases}
    a+3=-(1-b)\\b=10-a
\end{cases}\Rightarrow\quad
\begin{cases}
    a-b=-4\\ a+b=10
\end{cases}\]
所以
\[a=3,\qquad b=7\]
\end{solution}


























